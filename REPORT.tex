\documentclass{report}

\begin{document}

\title{Hardware Accelerators - Converting Code to Silicon}
\author{Jay Valentine}

\maketitle

\begin{abstract}
\end{abstract}

\chapter*{Introduction}

\chapter*{Literature Review}

\section{Dark Silicon}

Dark silicon is a phenomenon which has been observed as transistor density in microprocessors has increased.
It occurs as issues relating to energy efficiency and transistor utilization lead to a gap between the observed speedup and that
predicted by extrapolating from historic performance gains. In \cite{darksilicon}, it is predicted that with 22nm transistors,
21\% of the chip will be dark silicon, with this rising to 50\% at 8nm. This prediction shows that dark silicon will become a
serious limitation as transistor density grows, especially in areas where energy efficiency is a primary concern.

Dark silicon also becomes a limiting factor with manycore devices. Even when energy consumption is not a concern, the limited
parallelism of most applications results in a dark silicon gap when running with manycore devices. Again, \cite{darksilicon} shows
that beyond a certain number of cores the speedup achieved is negligible. This is another kind of dark silicon - the
underutilization in this case is not a result of energy concerns but is caused by the limited parallelism of the application being
unable to exploit all of the cores of a device.

\section{Energy Efficiency}

Many attempts at managing dark silicon for energy efficiency have been made. One approach, outlined in the GreenDroid project
\cite{greendroid}, is to replace areas of a chip that cannot be utilized due to dark silicon with specialized cores, designed
to perform common computations. In the GreenDroid architecture, these cores account for over 90\% of execution time,
leading to a significant reduction in energy consumption. This was aided by profiling of the Android platform, allowing the
identification of 'hot' (frequently-executed) portions of code to be offloaded to specialized logic. It was found that Android,
due to the organization of its components, was an ideal candidate for this approach.

Other attempts have been made at managing dark silicon in less specific domains. In \cite{arnone-thesis}, Arnone outlines a
method of generating accelerator cores for a stack architecture, resulting in both timing and power improvements. Two architectures
for generated cores are described: composite and wave-core. Composite cores are simple state machines with instructions mapped to
states. Composite cores attempt to optimise for logic area by reusing existing logic between states. Conversely, the wave-core
architecture avoids the reuse of logic between states, reducing power density at the cost of increased logic area (due to
potentially duplicated functionality). In both cases the resulting accelerator core is more energy-efficient than the
general-purpose processor is at the same task.

\section{Performance}

In \cite{high-performance-microarchitecture}, Razdan and Smith attempt to simplify the state machine model to reduce
synchronization complexities. They developed a toolchain which was able to extract instruction streams that could be 'outsourced'
to an accelerator core after code generation. These cores are not state machines but are instead streams of instructions translated into hardware, such that a previously multi-cycle operation can be executed in a single cycle. The instruction
stream that existed in the original application can then be replaced by a single instruction, \textit{expfu}, which triggers
the respective accelerator core and retrieves the result. As SoCs (as opposed to discrete units) become more prevelant,
this work shows that there may be a place for FPGAs in such a system.

\section{Architectural Support}

If accelerator-rich architectures are to surpass conventional single- or multi-core devices, the overheads associated with
the communication between accelerators and the general purpose CPUs need to be alleviated. In \cite{accelerator-rich-cmp},
an approach to this is outlined in which a hardware architecture and a lightweight interrupt system are developed.

\begin{thebibliography}{9}
\bibitem{darksilicon}
H. Esmaeilzadeh, E. Blem, R. St. Amant, K. Sankaralingam, D. Burger, "Dark Silicon and the End of Multicore Scaling", IEEE Micro, vol. 32, pp. 122-134, May-June 2012.

\bibitem{greendroid}
S. Swanson, M. B. Taylor, "Greendroid: Exploring the Next Evolution in Smartphone Application Processors", IEEE Communications Magazine, vol. 49, pp 112-119, April 2011.

\bibitem{arnone-thesis}
A. Arnone, "Feasibility of Accelerator Generation to Alleviate Dark Silicon in a Novel Architecture", Ph.D. thesis, Dept. Computer Science, University of York, York, June 2017.

\bibitem{high-performance-microarchitecture}
R. Razdan, M. D. Smith, "A High-Performance Microarchitecture with Hardware-Programmable Functional Units", Proceedings of MICRO-27 (The 27th Annual IEEE/ACM International Symposium on Microarchitecture), pp 172-180, Nov 1994.

\end{thebibliography}

\end{document}